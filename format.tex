% !Mode:: "TeX:UTF-8"
\setlength{\subfigbottomskip}{0pt}
\CTEXoptions[bibname={国内外主要参考文献}]
\CTEXsetup[name={,},number={}]{chapter}
\captionsetup{labelsep=space,font=small,justification=centering}
\arraycolsep=1.7pt
\graphicspath{{figures/}}
\renewcommand{\subcapsize}{\zihao{5}}
\renewcommand{\thesubfigure}{\alph{subfigure})}
\setcounter{secnumdepth}{4}
\newcommand{\pozhehao}{\raisebox{0.1em}{------}}
\titleformat{\chapter}{\center\zihao{-2}\heiti}{\chaptertitlename}{0.5em}{}
\titlespacing{\chapter}{0pt}{-4.5mm}{8mm}
\titleformat{\section}{\zihao{-3}\heiti}{\thesection}{0.5em}{}
\titlespacing{\section}{0pt}{4.5mm}{4.5mm}
\titleformat{\subsection}{\zihao{4}\heiti}{\thesubsection}{0.5em}{}
\titlespacing{\subsection}{0pt}{4mm}{4mm}
\titleformat{\subsubsection}{\zihao{-4}\heiti}{\thesubsubsection}{0.5em}{}
\titlespacing{\subsubsection}{0pt}{0pt}{0pt}
\makeatletter
\renewcommand\thesection{\@arabic \c@section} % 前面不带 thechapter
\makeatother

\theoremstyle{plain}
\theorembodyfont{\songti\rmfamily}
\theoremheaderfont{\heiti\rmfamily}
\newtheorem{definition}{\heiti 定义}
\newtheorem{example}{\heiti 例}
\newtheorem{algo}{\heiti 算法}
\newtheorem{theorem}{\heiti 定理}
\newtheorem{axiom}{\heiti 公理}
\newtheorem{proposition}{\heiti 命题}
\newtheorem{lemma}{\heiti 引理}
\newtheorem{corollary}{\heiti 推论}
\newtheorem{remark}{\heiti 注解}
\newenvironment{proof}{\noindent{\heiti 证明:}}{\hfill $ \square $ \vskip 4mm}
\theoremsymbol{$\square$}

% 定义页眉和页脚 使用fancyhdr 宏包
% \newcommand{\makeheadrule}{
% \rule[7pt]{\textwidth}{0.75pt} \\[-23pt]
% \rule{\textwidth}{0.75pt}}
% \renewcommand{\headrule}{
%     {\if@fancyplain\let\headrulewidth\plainheadrulewidth\fi
%      \makeheadrule}}
\fancypagestyle{plain}{%
\fancyhf{} % clear all header and footer fields
\fancyfoot[C]{\fontsize{9pt}{9pt}\selectfont\thepage} % except the center
\renewcommand{\headrulewidth}{0pt}
\renewcommand{\footrulewidth}{0pt}}
\pagestyle{plain}

\renewcommand{\CJKglue}{\hskip 0.56pt plus 0.08\baselineskip} %加大字间距,使每行33个字
\def\defaultfont{\renewcommand{\baselinestretch}{1.62}\normalsize\selectfont}
% 调整罗列环境的布局
\setitemize{leftmargin=3em,itemsep=0em,partopsep=0em,parsep=0em,topsep=-0em}
\setenumerate{leftmargin=3em,itemsep=0em,partopsep=0em,parsep=0em,topsep=0em}
\renewcommand{\theequation}{\arabic{equation}}
\renewcommand{\thetable}{\arabic{table}}
\renewcommand{\thefigure}{\arabic{figure}}

\makeatletter
\renewcommand{\p@subfigure}{\thefigure~}
\makeatother

\newcommand{\citeup}[1]{\textsuperscript{\cite{#1}}} % for WinEdt users

% 封面、摘要、版权、致谢格式定义
\makeatletter
\def\title#1{\def\@title{#1}}\def\@title{}
\def\titlesec#1{\def\@titlesec{#1}}\def\@titlesec{}
\def\affil#1{\def\@affil{#1}}\def\@affil{}
\def\subject#1{\def\@subject{#1}}\def\@subject{}
\def\researchdirection#1{\def\@researchdirection{#1}}\def\@researchdirection{}
\def\author#1{\def\@author{#1}}\def\@author{}
\def\bdate#1{\def\@bdate{#1}}\def\@bdate{}
\def\supervisor#1{\def\@supervisor{#1}}\def\@supervisor{}
\def\assosupervisor#1{\def\@assosupervisor{\textbf{副\hfill 导\hfill 师} & \rule[-4pt]{200pt}{1pt}\hspace{-326pt}\centerline{\textbf {#1}}\\}}\def\@assosupervisor{}
\def\cosupervisor#1{\def\@cosupervisor{\textbf{联\hfill 合\hfill 导\hfill 师} & \rule[-4pt]{200pt}{1pt}\hspace{-326pt}\centerline{\textbf {#1}}\\}}\def\@cosupervisor{}
\def\date#1{\def\@date{#1}}\def\@date{}
\def\stuno#1{\def\@stuno{#1}}\def\@stuno{}
% 定义封面
\ifxueweidoctor
\def\makecover{
    \thispagestyle{empty}
    \zihao{2}
		\renewcommand{\CJKglue}{\hskip 2pt plus 0.08\baselineskip}
    \centerline{\xingkai{中国科学技术大学}}
		\vspace{5mm}
		\centerline{\zihao{1}\songti\textbf{研究生学位论文开题报告}}
    \zihao{2}\vspace*{40mm}
		% \renewcommand{\CJKglue}{\hskip 2pt plus 0.08\baselineskip}
    % \centerline{\songti\textbf{题目:\rule[-4pt]{220pt}{1pt}\hspace{-220pt}\textbf\@title}}		
		% \centerline{\hspace{10em}\songti\textbf{\textbf\@titlesec}}		
    % \zihao{3}\vspace{4\baselineskip}
    \hspace*{-20pt}
    {\songti
	\renewcommand{\arraystretch}{1.1}
    \begin{tabular}{l@{}l}
    {论文题目}      & \rule[-4pt]{300pt}{1pt}\hspace{-390pt}\centerline{\@title}\\
    {}             & \rule[-4pt]{300pt}{1pt}\hspace{-390pt}\centerline{\@titlesec}\\
    {学生姓名}      & \rule[-4pt]{300pt}{1pt}\hspace{-390pt}\centerline{\@author}\\
    {学生学号}  & \rule[-4pt]{300pt}{1pt}\hspace{-390pt}\centerline{\@stuno}\\
    {指导老师}     & \rule[-4pt]{300pt}{1pt}\hspace{-390pt}\centerline{\@supervisor}\\
    {所在院系}   & \rule[-4pt]{300pt}{1pt}\hspace{-390pt}\centerline{\@affil}\\
    {学科专业}     & \rule[-4pt]{300pt}{1pt}\hspace{-390pt}\centerline{\@subject}\\
    {研究方向}     & \rule[-4pt]{300pt}{1pt}\hspace{-390pt}\centerline{\@researchdirection}\\
    {填表日期} & \rule[-4pt]{300pt}{1pt}\hspace{-390pt}\centerline{\@date}\\
    \end{tabular}\renewcommand{\arraystretch}{1}}
  \vfill
    \centerline{\zihao{-3}\kaishu{中国科学技术大学研究生院培养办公室}}
    \vspace{0.1\baselineskip}
    \centerline{\zihao{-3}\kaishu{二零零四年五月制表}}

%定义内封
    %新版本没有内封
\newpage
\thispagestyle{empty}
\begin{center}
 \heiti\zihao{2}{说\hspace{1em}明}
\end{center}
\vspace*{40pt}
   {\kaishu\zihao{3}

   \noindent 1.	抓好研究生学位论文开题报告工作是保证学位论文质量的一个重要环节。为加强对研究生培养的过程管理,规范研究生学位论文的开题报告,特印发此表。
   
   \vspace*{20pt}
   
   \noindent 2.	研究生一般应在课程学习结束之后的第一个学期内主动与导师协商,完成学位论文的开题报告。
   
   \vspace*{20pt}
   
   \noindent 3.	研究生需在学科点内报告,听取意见,进行论文开题论证。
   
   \vspace*{20pt}
   
   \noindent 4.	研究生论文开题论证通过后,在本表末签名后将此表交所在学院教学办公室备查。

   }
	\renewcommand{\arraystretch}{1}
    \clearpage
}
\fi

\ifxueweimaster
\def\makecover{
    \thispagestyle{empty}
    \zihao{-2}\vspace*{10mm}
		\renewcommand{\CJKglue}{\hskip 2pt plus 0.08\baselineskip}
    \centerline{\kaishu\textbf{中国科学技术大学}}
		
		\vspace{10mm}
		\centerline{\zihao{2}\songti\textbf{硕士学位论文开题报告}}

		\renewcommand{\CJKglue}{\hskip 0pt plus 0.08\baselineskip}
\vspace{30pt}
\zihao{-2}
\begin{center}\songti\textbf{题~目:\@title}\end{center}
\vspace{30pt}
    \zihao{3}
    \hspace*{68pt}{\songti
	\renewcommand{\arraystretch}{1.3}
    \begin{tabular}{l@{}l}
    \textbf{院\hfill (系)}   & \rule[-4pt]{200pt}{1pt}\hspace{-326pt}\centerline{\textbf\@affil}\\
    \textbf{学\hfill 科}     & \rule[-4pt]{200pt}{1pt}\hspace{-326pt}\centerline{\textbf\@subject}\\
    \textbf{导\hfill 师}     & \rule[-4pt]{200pt}{1pt}\hspace{-326pt}\centerline{\textbf\@supervisor}\\
    \@assosupervisor
	\@cosupervisor
    \textbf{研\hfill 究\hfill 生}      & \rule[-4pt]{200pt}{1pt}\hspace{-326pt}\centerline{\textbf\@author}\\
    \textbf{学\hfill 号}  & \rule[-4pt]{200pt}{1pt}\hspace{-326pt}\centerline{\textbf\@stuno}\\
    \textbf{开题报告日期} & \rule[-4pt]{200pt}{1pt}\hspace{-326pt}\centerline{\textbf\@date}\\
    \end{tabular}
		\renewcommand{\arraystretch}{1}}
	\vfill
    \centerline{\songti\textbf{研究生院培养处制}}

%%定义内封
\newpage
\thispagestyle{empty}
\zihao{5}\vspace*{2em}
\begin{center}
  \heiti\zihao{3}说\hspace{3em}明
\end{center}
\vspace*{40pt}
	\renewcommand{\arraystretch}{1.25}
    {\songti\zihao{5}
    \hangindent=2em
	\noindent 一、开题报告应包括下列主要内容:
    \begin{enumerate}[leftmargin=36pt]
	\item 课题来源及研究的目的和意义;
	\item 国内外在该方向的研究现状及分析;
	\item 主要研究内容;
	\item 研究方案及进度安排,预期达到的目标;
	\item 为完成课题已具备和所需的条件和经费;
	\item 预计研究过程中可能遇到的困难和问题,以及解决的措施;
	\item 主要参考文献。
    \end{enumerate}
    \noindent 二、对开题报告的要求
	\begin{enumerate}[leftmargin=36pt]
	\item 开题报告的字数应在~5000~字以上;
	\item 阅读的主要参考文献应在~20~篇以上,其中外文资料应不少于三分之一。硕士研究生应在导师的指导下着重查阅近年内发表的中、\hspace{-1pt}外文期刊文章。\hspace{-1pt}本学科的基础和专业课教材一般不应列为参考资料。
    \end{enumerate}
    \noindent 三、开题报告时间应最迟不得超过第三学期的第三周末。

    \hangindent=2em\noindent 四、如硕士生首次开题报告未通过,\hspace{-2pt}需在一个月内再进行一次。\hspace{-3pt}若仍不通过,\hspace{-2pt}则停止硕士论文工作。

    \noindent 五、此表不够填写时,可另加附页。

\hangindent=2em\noindent 六、开题报告进行后,此表同硕士学位论文开题报告评议结果存各系(院)研究生秘书书处,以备研究生院及所属学院进行检查。

    }
	\renewcommand{\arraystretch}{1}
    \clearpage
}
\fi
\makeatother
